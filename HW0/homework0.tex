\documentclass[twoside,11pt]{homework}

\coursename{COMS/CSOR 4231, Analysis of Algorithms I} 

\studname{Jing Qian}    % YOUR NAME GOES HERE
\studmail{jq2282@columbia.edu}% YOUR UNI GOES HERE
%\hwNo{0}                   % THE HOMEWORK NUMBER GOES HERE
\date{\today} % DATE GOES HERE

% Uncomment the next line if you want to use \includegraphics.
%\usepackage{graphicx}

\begin{document}
\maketitle



\section*{Homework 0}
\textbf{1. Department:} Computer Science
\\\\
\textbf{2. Program:} Master of Science
\\\\
\textbf{3. Year:} 2018 Fall
\\\\
\textbf{4. List relevant background courses and where you have taken them:}
\\
Algebra and Geometry, Advanced Mathematical Analysis, Practical Fundamentals of Computer, C Language Programming, Probability Theory and Mathematical Statistics, Methods of Mathematics and Physics, Mathematical Modeling, Group Theory in Physics.
\\
From Harbin Institute of Technology in China.
\\\\
\textbf{5. How familiar are you with the following topics (None, Somewhat, Very):
\\
\indent  1. Recurrences:} Very \\
\indent \textbf{2. Proof by induction:} Very \\
\indent  \textbf{3. Big-O notation:} Very \\
\indent  \textbf{4. Probability:} Very \\
\indent  \textbf{5. Balanced search trees:} Somewhat \\ 
\indent  \textbf{6. Graph algorithms:} None \\
\indent  \textbf{7. NP-completeness:} None
% YOUR SOLUTION GOES HERE
\\\\
\textbf{6. Why are you taking this course?}\\
There are two main reasons I take this course:\\
1. In my previous study majoring in Physics, I wrote some code by myself which I found long-winded and slow. I want to make my code efficient and concise.\\
2. Many algorithms themselves are interesting to learn. I appreciate and enjoy the  human wisdom and aesthetics in algorithms.
% SOME EXAMPLE LATEX CODE BELOW (DON'T INCLUDE IN YOUR ACTUAL SUBMISSION!)
%Examples of blackboard and calligraphic letters: $\bbR^d \supset
%\bbS^{d-1}$, $\cC \subset \cB$.
%Examples of bold-faced letters (perhaps suitable for matrix and
%vectors):
%\begin{equation}
%  L(\vx,\vlambda) = f(\vx) - \innerprod{\vlambda,\vA\vx-\vb} .
%  \label{eq:lagrangian}
%\end{equation}
%\newcommand\var{\ensuremath{\operatorname{var}}}%
%Example of a custom-defined math operator:
%\[
%  \var(X) = \bbE X^2 - (\bbE X)^2 .
%\]
%Example of references: the Lagrangian is given in
%Eq.~\eqref{eq:lagrangian}, and Theorem~\ref{thm:euclid} is
%interesting.
%Example of adaptively-sized parentheses:
%\[
%  \left(\prod_{i=1}^n x_i\right)^{1/n}
%  + \left(\prod_{i=1}^n y_i\right)^{1/n}
%  \leq
%  \left(\prod_{i=1}^n (x_i + y_i)\right)^{1/n}
%  .
%\]
%Example of aligned equations:
%\begin{align}
%  \Pr(X = 1 \,|\, Y = 1)
%  & = \frac{\Pr(X = 1 \,\wedge\, Y = 1)}{\Pr(Y = 1)}
%  \notag \\
%  & =
%  \underbrace{
%    \frac{\Pr(Y = 1 \,|\, X = 1) \cdot \Pr(X = 1)}{\Pr(Y = 1)}
%  }_{\text{Usual expression for Bayes' rule}}
%  .
%  \label{eq:bayes-rule}
%\end{align}
%Example of a theorem:
%\begin{theorem}[Euclid]
%  \label{thm:euclid}
%  There are infinitely many primes.
%\end{theorem}
%\begin{proof}[Euclid's proof]
%  There is at least one prime, namely $2$.
%  Now pick any finite list of primes $p_1, p_2, \dotsc, p_n$.
%  It suffices to show that there is another prime not on the list.
%  Let $p := \prod_{i=1}^n p_i + 1$, which is not any of the primes on
%  the list.
%  If $p$ is prime, then we're done.
%  So suppose instead that $p$ is not prime.
%  Then there is prime $q$ which divides $p$.
%  If $q$ is one of the primes on the list, then it would divide $p -
%  \prod_{i=1}^n p_i = 1$, which is impossible.
%  Therefore $q$ is not one of the $n$ primes in the list, so we're
%  done.
%\end{proof}
%
%Here is a centered table.
%\begin{center}
%  \begin{tabular}{c||c|c|c|c}
%    & A
%    & B
%    & C
%    & D \\
%    \hline
%    \hline
%    $1$
%    & entries
%    & in
%    & a
%    & table
%    \\
%    \hline
%    $2$
%    & more
%    & entries
%    & more
%    & entries
%  \end{tabular}
%\end{center}

%Here is a centered figure. You'll need hw0.pdf in the same path.
%\begin{center}
%  \includegraphics[height=0.3\textheight]{hw0.pdf}
%\end{center}
%
%\section*{Problem 2}
%
%% YOUR SOLUTION GOES HERE
%
%\section*{Problem 3}
%
%% YOUR SOLUTION GOES HERE
%
%\section*{Problem 4}
%
%% YOUR SOLUTION GOES HERE
%
%\section*{Problem 5}
%
%% YOUR SOLUTION GOES HERE

\end{document} 
