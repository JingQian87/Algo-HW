\documentclass[twoside,11pt]{homework}
\usepackage{color}
\usepackage{listings}
\usepackage{graphicx} 

\coursename{CSOR 4231: Analysis of Algorithms I} 

\studname{Jing Qian}    % YOUR NAME GOES HERE
\studmail{jq2282}% YOUR UNI GOES HERE
\hwNo{5}                   % THE HOMEWORK NUMBER GOES HERE
%\date{\today} % DATE GOES HERE


\begin{document}
\maketitle

\section*{Problem 1}
\subsection*{a.}
We could prove this by contradition.
Assume a bipartite graph contains a cycle $C$ of odd lenth.
Let $v_1, v_2, \cdots, v_k$ be all the nodes listed in the cycle $C$ and $k$ is an odd number.
Then there is an edge between $v_1$ and $v_2$, an edge between $v_2$ and $v_3$, $\cdots$, and an edge between $v_k$ and $v_1$ because $C$ is a cycle.

Since it is a bipartite graph, every edge in the bipartite graph connects a node of $N_1$ with a node of $N_2$.
We could assume that all the odd nodes are in the subset $N_1$ and all the even nodes are in the subset $N_2$ (Similar provement works if odd nodes are in $N_2$ while even nodes are in $N_1$).
In other words, $v_1, v_3, \cdots v_k \in N_1$ and $v_2, v_4, \cdots v_{k-1} \in N_2$.

Above we show that modes $v_1$ and $v_k$ are in the same subset and there is an edge between $v_1$ and $v_k$, which disallow the property of bipartite graph that every edge connects nodes from different subsets.
Our assumption that a bipartite graph contains a cycle $C$ of odd lenth is wrong.
So if a graph contains a cycle of odd length then it is not bipartite.

\subsection*{b.}
We could solve this problem using the property of  bipartite that every edge connects nodes from different subsets and find the cyle of odd length according to Part a.
With a modified BFS, let nodes with the odd depth be in subset $N_1$ and nodes with even depth  be in subset $N_2$ and check whether there is an edge connecting two nodes in the same subset.

If there is an edge connecting two nodes $u$ and $v$ from the same subset, we trace back the parents of these two nodes in iteration and since it is a cycle, these two nodes would have a common ancestor $ca$. 
The cycle $u, \cdots, ca, \cdots, v, u$ would have an odd length and is what we want.
If we scan through all nodes and found no cycle with odd length, the graph would be bipartite and we separate nodes into the subset $N_1$ and $N_2$ according to the parity of their depth.

The code is as following:
%%%%%%%%%%%%%%%%%%CODE%%%%%%%%%%%%%%%%%%
\begin{colorboxed}
\begin{lstlisting}[language={[ANSI]C},numbers=left,numberstyle=\tiny, frame=single, title=Check\_Bipartite, breaklines=true,
   rulesepcolor=\color{red!20!green!20!blue!20},
   keywordstyle=\color{blue!70!black},
   commentstyle=\color{blue!90!},
   basicstyle=\ttfamily]
CHECK_BIPARTITE(G)
Starting at any node s: 
d[s] = 0
p[s] = NONE
Q = {s}
for each v in N-{s}: #N is the set of all nodes in G
    d[v] = Infinity
    p[v] = NONE
while Q not empty:
    u = Dequeue(Q)
    for each v in Adj[u]:
        if d[v] = Infinity:
            d[v] = d[u]+1
            p[v] = u
            Enqueue(Q, v)
        else if d[v] mod 2 == d[u] mod 2:
            C = {} #C is the cycle of odd lenth
	    while v != u:
	        Enqueue(C, u, v)
	        u = p[u]
	        v = p[v]
	    Enqueue(C, u)
	    Exit
N1 = {}
N2 = {}	    
for each v in G:
    if d[v] mod 2 == 1:
        N1 = N1 + {v}
    else:
        N2 = N2 + {v}
\end{lstlisting}
\end{colorboxed}   
%%%%%%%%%%%%%%%%%%CODE%%%%%%%%%%%%%%%%%%
The time complexity of this algorithm is O(n+e). 
The initialization (line 48-54) takes O(n) time.
The while loop (line 55-69) takes O(e) time.
If the graph is bipartite, the separation (line 72-76) takes O(n) time.
In total, O(n+e). 

\subsection*{c}
Let's assume all the 1-variables are nodes belong to the same subset $N_1$ and all the 0-variables are nodes belong to the same subset $N_2$.
Then the inequality $x_i \neq x_j$ is the edge between $x_i \in N_1$ and $x_j \in N_2$ or reverse.
We could abstract this problem as an bipartite graph problem and use the algorithm in Part 1.b.
If the graph is not bipartite, there are edges between nodes in the same subset.
In other words, for $x_i \neq x_j$, $x_i$ and $x_j$ both are 0 (or 1), which is impossible to have a solution.
If the graph is bipartite, there are no edges between nodes in the same subset and all the constraints are consistent.
We could separate the variables into two subsets and that is the solution we want.
The time complexity is the same with that in Part 1.b., which is O(#nodes + #edges).
So O(n+m)


%%%%%%%%%%%%%%%%%%%%%
\section*{Problem 2}
Dijkstra's algorithm outputs the minimum weight path of graph $G$ from node $s$ to $t$.
To choose the minimum weight path with fewest number of edges, we only need to augment an variable $e$ to keep record of the number of edges in the path and choose the minimal $e$ when there is a tie between different path with the same weight.

The code is as following:
%%%%%%%%%%%%%%%%%%CODE%%%%%%%%%%%%%%%%%%
\begin{colorboxed}
\begin{lstlisting}[language={[ANSI]C},numbers=left,numberstyle=\tiny, frame=single, title=ModifiedDijkstra, breaklines=true,
   rulesepcolor=\color{red!20!green!20!blue!20},
   keywordstyle=\color{blue!70!black},
   commentstyle=\color{blue!90!},
   basicstyle=\ttfamily]
MODIFIED_DIJKSTRA(G, s, t)
d[s] = 0
p[s] = NONE
e[s] = 0
for each v in N-{s}: #N is the set of all nodes in G
    d[v] = Infinity
    p[v] = NONE
    e[v] = Infinity
Q = N
while Q not empty:
    u = Extract-Min(Q)
    for each v in Adj[u]:
        if d[v] > d[u] + w(u, v):
            d[v] = d[u] + w(u, v)
            p[v] = u
            e[v] = e[u] + 1
        else if d[v] == d[u] + w(u, v) and e[v] > e[u] + 1:
            p[v] = u
            e[v] = e[u] + 1     
\end{lstlisting}
\end{colorboxed}   
%%%%%%%%%%%%%%%%%%CODE%%%%%%%%%%%%%%%%%%
According to the code above, we only add one variable and an \textit{else if} in every iteration of the while loop.
So the time complexity would be the same with original Dijkstra algorithm.
And we still output the minimum weight path, but choose the one with the minimum number of edges when there are multiple minimum weight paths.


%%%%%%%%%%%%%%%%%%%%%%%%%%%%%%
\section*{Problem 3}
\subsection*{a.}
First, if there is path from $x_v$ to $x_u$, we have $x_v \le x_i < x_j, \cdots, \le x_u$.
Hence $x_v \le x_u$, which contradicts the assumption that set $C$ contains a strict inequality $x_u < x_v$.
So If the set $C$ contains a strict inequality $x_u < x_v$ such that the graph $G$ has a path from $x_v$ to $x_u$, it is not consistent.

Second, if the set $C$ is not consistent, there must be a cycle in the graph $G$.
If all edges in the cycle are non-strict inequality, i.e. $x_u \le x_v$, we still have a consistent set $C$ because all the nodes could be the same.
But if one edge in the cycle is strict inequality, set $C$ is not consistent.
Let's say the edge is $x_u < x_v$, then the rest edges of this cycle becomes a path from $x_v$ to $x_u$.
So if the set $C$ is not consistent, it must contain a strict inequality $x_u < x_v$ such that the graph $G$ has a path from $x_v$ to $x_u$.

In conclusion, the set $C$ is not consistent if and only if it contains a strict inequality $x_u < x_v$ such that the graph $G$ has a path from $x_v$ to $x_u$.

\subsection*{b.}
We could use the property in Part 3.a. to solve this problem.
If there is  a strict inequality $x_u < x_v$ such that the graph $G$ has a path from $x_v$ to $x_u$, set $C$ is not consistent.
We could use a modified toplogical sort to solve this problem.
We sort using DFS and color every nodes to show whether they are visted or not.
If expaning the node $u$ and find its adjacent node $v$ has been reached, we check whether the constrain between $u$ and $v$ is a strict inequality.
If it is,  $C$ is not consistent.
If the constrain between $u$ and $v$ is not strict, we go to the other node $w$ of the back edge $(u, w)$ and check whether there is a strict inequal edge in the path $(w, u)$. 
If there is, $C$ is not consistent.
If there isn't, go to the other adjacent nodes of $v$ and continue.
This algorithm still takes $O(n+m)$ like normal DFS because we only add $O(m)$ edge check.



\subsection*{c}
We could use the algorithm in Part 3.b to solve this problem.
If the set of inequalities has a solution over the positive integers, the set is consistent.
We could use the algorithm in Part 3.b. to output a path of all nodes in the graph G: $x_1' \le x_2'< \cdots, \le x_n'$.
Start with the minimal variable $x_1'$, we assign it as 1.
If $x_1' \le x_2'$, $x_2' = 1$.
Or if $x_1' < x_2'$, $x_2' = 2$.
To generalize, if we got the value of $x_i'$, if $x_i' \le x_j'$, $x_j' = x_i'$;
if $x_i' < x_j'$, $x_j' = x_i' + 1$.
The assignment takes an extra O(m) time than the algorithm in Part 3.b. because we have to check all the edges in the path.
But in total, the time complexity remians O(n+m).

%%%%%%%%%%%%%%%%%%%%%%%%%%%%%%%%
\section*{Problem 4}
\subsection*{a.}
First, let's assume that the root $R$ of $G_{\pi}$ is not an articulation point of $G$ while it has at least two children in $G_{\pi}$: $C_1$ and $C_2$.
Since $R$ is not articulation point, according to the definition, $C_1$ and $C_2$ are connected in a path without $R$.
Since it is a DFS, if $C_2$ is reachable to $C_1$ without going through $R$, $C_1$ must be an ancestor
of $C_2$.
However, we know that the only ancestor of $C_2$ is the root $R$, which contradicts with the induction that $C_1$ is an ancestor of $C_2$.
So, if  the root $R$ has at least two children in $G_{\pi}$, it is an articulation point of $G$.

Second, if the root $R$ of $G_{\pi}$ is an articulation point of $G$ while it has only one or no child.
If $R$ has no child, $G$ contains only one node.
When remove $R$, there is nothing inside $G$ and hence no disconnection.
So $R$ is not an articulation point if it has no child.
If $R$ has one child, when remove $R$, the child of $R$ becomes the root of the new DFS tree and the graph $G$ remains connected.
So $R$ is not an articulation point if it has one child.

In conclusion,  the root of $G_{\pi}$ is an articulation point of $G$ if and only if it has at least two children in $G_{\pi}$.

\subsection*{b.}
First, if $v$ has a child $s$ such that there is no back edge from $s$ or any descendent of $s$ to a proper ancestor of $v$, when we remove $v$, the child $s$ has no connection with the ancestors of $v$ and the graph $G$ become disconnected.
So $v$ is the articulation point.

Second, assuming $v$ is the articulation point and it has a child $s$ such that there is a back edge from $s$ or any descendent of $s$ to a proper ancestor of $v$, let's say there is a back edge $E(s, w)$ in which $w$ is a proper ancestor of $v$ (Similar provement if the back edge is $E(t, w)$ in which $t$ is a descendent of $s$).
When we remove $v$, the child $s$ and its descendents are still connected with $w$ and all the descendents or ancestors of $w$.
$G$ still remains connected, which contradicts with the assumption that $v$ is the articulation point.

In conclusion, $v$ is the articulation point if and only if it has a child $s$ such that there is no back edge from $s$ or any descendent of $s$ to a proper ancestor of $v$.

\subsection*{c.}
According to the definition of $v.low$, the $low$ property of node $v$ is determined by itself and its descendant.
So we could use a bottom up method to compute $v.low$ all vertices.

First, let's consider the situation that $v$ is a leaf.
If there is no back edge for $v$ or we don't take $v$ as a descendant of itself, $v.low = v.d$.
If there is a back edge to a proper ancestor $w$of $v$ and we take $v$ as a descendat of itself, $v.low = w.d$.

Then, let's consider the situation that $v$ is an intermediate or top node.
The $v.low$ is the minimum of $v.d$ and the $c.low$ of all its childrens $c$.
We iterate from bottom leaves to top root and calculate the $v.low$ for all vertices $v \in V$.
Because this algorithm goes through all the edges and only once, the time complexity if $O(E)$.

\subsection*{d.}
We could combine the algorithm in Part 4.c. and the property in Part 4.b. to solve this problem.

The node $v$ is the articulation point if and only if it has a child $s$ such that there is no back edge from $s$ or any descendent of $s$ to a proper ancestor of $v$.
If the node $v$ has a child $s$ such that there is a back edge from $s$ or any descendent of $s$ to a proper ancestor of $v$, the $v.low$ would be $w.d$ which is smaller than $w.d$.

So we could modify the algorithm in Part 4.c.
For every $v$, after we compute the $v.low$, we check whether it equals to $v.d$.
If $v.low = v.d$,  it has no child $s$ such that there is a back edge from $s$ or any descendent of $s$ to a proper ancestor of $v$.
Then $v$ is an articulation point.
Else, $v$ is not an articulation point.
Since this modified algorithm only adds one equality check in every iteration of the original one, the time complexity remains the same, $O(E)$.
So we could get all articulation points in $O(E)$ time.


%%%%%%%%%%%%%%%%%%%%%%%%%%%%%%%%
\section*{Problem 5}
\subsection*{a.}
Let's assume there are $t$ edges in the minimum spanning tree $T$while  not  in the spanning tree $T_k$ computed by Kruskal's algorithm , i.e. $e_i \notin T_k$ and $e_i \in T$ ($i$ from 1 to $t$). 

Take the edge $e_1 \notin T_k$,  $e_1 \cup T_k$ must generate a cycle $C$ since $T$ is a spanning tree.
According to the Kruskal's algorithm, every edge in the cycle $C$ is smaller or equal to the weight of $e_1$. 
Let's assume there's an edge $f_1$ in the cycle $C$ which $f_1 \in T_k$ and $f_1 \notin T$.
We could construct a new spanning tree $T_1 = T_k \cup e_1 - f_1$.
Since the weight of $f_1$ is smaller or equal to that of $e_1$, the weight of $T_1$ is larger or equal to that of $T_k$, $T_k \le T_1$.

Using the similar method, we could generate $T_2 = T_1 \cup e_2 - f_2$ and $T_1 \le T_2$,  $T_{j+1} = T_j \cup e_{j+1} - f_{j+1}$ and $T_j \le T_{j+1}$.

Finally, all the $t$ different edges are modified in $T_k$ and $T_k$ becomes $T$ and we have:
$T_k \le T_1 \le T_2 \cdots \le T_j \le T_{j+1} \le T$.
Since $T$ is the minimum spanning tree, the only way the inequality above holds is that every spanning tree has the same weight: $T_k = T_1 = T_2 \cdots = T_j = T_{j+1} = T$.
And all the deleted $f_j$ equal to their corresponding added $e_j$.
So $T_k$ is minimum spanning tree.



\subsection*{b.}
If we decrease the weight of one of the edges that is not in $T$, it may or may not affect the minimum spanning tree $T$.
Let's call this modified edge $e(u, v)$ connecting two nodes $u$ and $v$.
Since $e$ is not in $T$, there must be a path from node $u$ to $v$ in $T$.
Then we could check whether there is an edge in $T$ has a larger weight than $e$.
If there is an edge $f(i, j)$ in the path $u - \cdots - i - j \codts -v$ having larger weight than $e(u, v)$, we remove the edge $f(i, j)$ from $T$, union the edge $e(u, v)$, and have a new minimum spanning tree $T^*$ which has smaller weight than $T$.
If there is no edge with larger weight than $e$, $T$ remains the same.

In this algorithm, we take $O(n)$ time to find the node $u$ because $T$ is given by adjacency list representations.
Then we follow the path $u$ to $v$ in $T$ and check the weight of every edge along this path, which also takes $O(n)$ time.
If we modify $T$, the union between $T$ and new edge $e$ takes $O(1)$ time.
So in total, the time complexity of the algorithm is $O(n)$.


\end{document}
